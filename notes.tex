\documentclass[10pt,oneside,a3paper]{amsbook}

\title{Real and linear analysis}
\author{Course notes based on material from\\``Measure theory'' by Terence Tao, and\\``Real analysis'' by Bruckner, Bruckner, and Thompson}

\usepackage[margin=1in]{geometry}
\usepackage{amssymb}
\usepackage{setspace}\onehalfspacing\raggedbottom
\renewcommand{\labelitemi}{$\circ$}
\renewcommand{\labelenumi}{(\alph{enumi})}
\renewcommand{\chaptername}{Part}
\renewcommand{\thechapter}{\Roman{chapter}}
\usepackage{remreset}
\makeatletter\@removefromreset{section}{chapter}\makeatother
\usepackage{etoolbox}
% \makeatletter
% \pretocmd{\@seccntformat}{\S}{}{}      % mark sec numbers
% \patchcmd{\tocsection}{#2.}{\S#2.}{}{} % mark sec numbers in toc
% \apptocmd{\tocsection}{\dotfill}{}{}   % dots in toc
% \patchcmd{\@maketitle}{\newpage}{}{}{} % no blank page after title
% \makeatother
\usepackage[linktoc=all]{hyperref}
\usepackage{tikz}
\pagestyle{empty}

\newcommand{\set}[1]{\left\{\,#1\,\right\}}
\renewcommand{\setminus}{\smallsetminus}
\renewcommand{\triangle}{\bigtriangleup}
\newcommand{\NN}{{\mathbb N}}
\newcommand{\ZZ}{{\mathbb Z}}
\newcommand{\QQ}{{\mathbb Q}}
\newcommand{\RR}{{\mathbb R}}
\newcommand{\CC}{{\mathbb C}}
\DeclareMathOperator{\len}{len}
\DeclareMathOperator{\vol}{vol}
\DeclareMathOperator{\dom}{dom}
\DeclareMathOperator{\rng}{rng}
\DeclareMathOperator{\id}{id}
\newcommand{\sint}{\text{s}\kern-3pt\int}
\newcommand{\lint}{\underline\int}
\newcommand{\ovint}{\overline\int}

\theoremstyle{definition}
\newtheorem{exerc}{Exer}[section]
\swapnumbers
\theoremstyle{plain}
\newtheorem{thm}{Thm}[section]
\newtheorem{cor}[thm]{Cor}
\newtheorem{lem}[thm]{Lemma}
\newtheorem{prop}[thm]{Prop}
\theoremstyle{definition}
\newtheorem{defn}[thm]{Def}
\theoremstyle{remark}
\newtheorem{rem}[thm]{Remark}
\newtheorem{example}[thm]{Example}
\newtheorem*{notes}{Notes and further reading}
\newtheorem*{reading}{Reading}
\numberwithin{equation}{section}
\numberwithin{figure}{section}
\renewcommand{\theequation}{\arabic{section}.e\arabic{equation}}
\renewcommand{\thefigure}{\arabic{section}.f\arabic{figure}}

\begin{document}

% \maketitle

% \tableofcontents
% \newpage

% %%%%%%%%%%%%%%%%%%%%%%%%%%%%%%%%%%%%%%%%%%%%%%%%%%
% %%%%%%%%%%%%%%%%%%%%%%%%%%%%%%%%%%%%%%%%%%%%%%%%%%
% \chapter{Measure theory}
% %%%%%%%%%%%%%%%%%%%%%%%%%%%%%%%%%%%%%%%%%%%%%%%%%%
% %%%%%%%%%%%%%%%%%%%%%%%%%%%%%%%%%%%%%%%%%%%%%%%%%%
% 
%%%%%%%%%%%%%%%%%%%%%%%%%%%%%%%%%%%%%%%%%%%%%%%%%%
\section{The measure problem}
%%%%%%%%%%%%%%%%%%%%%%%%%%%%%%%%%%%%%%%%%%%%%%%%%%

\begin{reading}
  Tao, \S1.1 introduction, and \S1.2.3.
\end{reading}

Before discussing the measure problem, let's talk intuitively about what we mean by ``measure.'' In this area of mathematics, measure is a number assigned to a set which represents its size. Of course the term ``size'' also has many meanings. Our concept of measure can accommodate many different types of sizes, such as: length, area, volume, mass, and even probability. On the other hand, some other types of size are not usually associated with the mathematical concept of measure, such as: cardinality, diameter, and density.

The mathematical concept of a measure is thus beginning to seem geometric. But considering our models of sets and spaces, in which coordinate axes are indexed by the infinitesimal points of the real number line, measure really turns out to be an analytic concept. (In particular, this means lots of $\epsilon$'s will show up in our studies!)

For a simple set, it may be easy to decide what its measure should be. For example, if we use the term measure to mean length, then the measure of the interval $[4,7]$ should be $3$. But for a more complicated set, the decision may not be so easy. If you have seen the construction of the Cantor set, think about how you would measure the length of that!

Thus we arrive at the ``measure problem,'' which asks whether it is even possible to find a function which adequately measures subsets of the real line $\RR$. Of course it is necessary to say what is considered adequate. The classical version of the measure problem proposed the three properties below. Formally, the measure problem asks: Does there exist a measure function $m$ which assigns to each subset $A\subset\RR$ a value $m(A)\in[0,\infty]$ satisfying:
\begin{enumerate}
\item (normality) $m(I)=$ the length of $I$ for every interval $I$;
\item (translation-invariance) $m(x+A)=m(A)$ for every $A$; and
\item (countable additivity) $m(\bigcup_{n=1}^{\infty} A_n)=\sum m(A_n)$ for every seqence of pairwise disjoint sets $A_n$.
\end{enumerate}

Perhaps surprisingly, no such measure function $m$ exists! While properties~(a)--(c) seem very natural, the three items unfortunately turn out to be mutually inconsistent.

\begin{thm}[Vitali]
  There exists a set $A\subset\RR$ such that no measure can be assigned to $A$ consistently with (a)--(c).
\end{thm}

\begin{proof}
  Rather than work on $\RR$, we will work on the half-open unit interval $[0,1)$ with the addition operation taken modulo $1$. This is ok, since if there is a measure $m$ on all subsets of $\RR$, then by properties~(b) and~(c), $m$ restricts to a measure on subsets of $[0,1)$ which satisfies property~(b) with respect to addition modulo $1$.

  Now let $\QQ_1$ denote the rationals of $[0,1)$, that is, $\QQ_1=\QQ\cap[0,1)$, and consider the collection of additive cosets of $\QQ_1$ inside $[0,1)$. The cosets are of the form $a+\QQ_1$ where again addition is interpreted modulo $1$. We now let $A\subset[0,1)$ denote a system of coset representatives for this collection.

  Now every number in $[0,1)$ can be written uniquely as $a+q$ for $a\in A$ and $q\in\QQ_1$. This means that the collection of translates of $A$ by elements $q\in\QQ_1$ covers all of $[0,1)$. In particular, by (a) the measure of $\bigcup_{q\in\QQ_1}(A+q)$ is exactly $1$.

  On the other hand, by (b) and (c) we have that
  \[m\left(\bigcup\nolimits_{q\in\QQ_1}(A+q)\right)
  =\sum\nolimits_{q\in\QQ_1}m(A+q)=\sum\nolimits_{q\in\QQ_1}m(A)
  \]
  By the previous paragraph, the left-hand side of the above equation is $1$. On the other hand the right-hand side is an infinite sum of some nonnegative constant, and hence must be either $0$ or $\infty$. This is a contradiction!
\end{proof}

We remark that it is possible to modify the argument to apply directly to a measure on $\RR$ rather than going via the unit interval with addition modulo $1$. See Tao for this version.

The lesson is that we must weaken our demands on a measure $m$. Dropping condition (a) can lead to trivial measures. Dropping condition (b) takes away the geometric aspects of the measure, and leads to interesting set-theoretic questions and constructions. Weakening condition (c) to finite additivity leads to interesting solutions, but only in dimensions $\leq2$. (In dimensions $\geq3$ the Banach--Tarski paradox again gives a contradiction.)

Yet the simplest path forward (and the one that we take) is to drop the tacit condition that \emph{every} set be measurable. The set $A$ constructed in Vitali's proof is very artificial and isn't likely to occur in any of the most common analytical applications (see the notes below). We want to be excused from the burden of deciding the measure of the set $A$. This means we need to figure out what sets we will measure, and what sets we will not measure. In the end, our measure function $m$ will have a domain which is a proper subset of $\mathcal P(\RR)$, but which still contains a rich collection of sets. And the measure will satisy properties (a)--(c) as long as they are applied to the sets in the domain of $m$.

Of course we are also interested in the measure problem for subsets $\RR^n$. It can be formulated in just the same way, with condition (a) replaced by the condition that the measure of a box is equal to its volume. And a Vitali-type result can also easily be established for this version of the measure problem.

In the next section, we will begin this process by taking a step backwards and build measures with much smaller domains, and satisfying just fragments of (a)--(c).

\begin{notes}
  The proof of Vitali's theorem requires the Axiom of Choice. Specifically, it is needed to find a system of coset representatives for an uncountable collection. Solovay showed that the use of AC is essential, and that it is consistent with $\neg$AC that there is a measure function $m$ on all subsets of $\RR$.
\end{notes}

\begin{exerc}
  Show that the properties (a)--(c) of a measure imply finite addivitity: If $A$ and $B$ are disjoint then $m(A\cup B)=m(A)+m(B)$.
\end{exerc}

\begin{exerc}
  Show that the properties (a)--(c) of a measure imply the inclusion--exclusion principle: For any sets $A,B$ we have $m(A\cup B)+m(A\cap B)=m(A)+m(B)$.
\end{exerc}

\begin{exerc}
  Complete the details of the proof that if there is a measure on $\RR$ satisfying properties (a)--(c), then there is a measure on $[0,1)$ (with additional modulo $1$) satisfying properties (a)--(c).
\end{exerc}

\begin{exerc}
  If $A$ is a bounded set of real numbers, the supremum $\sup(A)$ is the least upper bound of $A$, and the infemum $\inf(A)$ is the greatest lower bound of $A$. Show that $s=\sup(A)$ if and only if:
  \begin{itemize}
    \item for all $a\in A$ we have $a\leq s$, and;
    \item for all $\epsilon>0$, there exists $a\in A$ such that $s-a<\epsilon$.
  \end{itemize}
  Formulate and prove the analogous statement for the infemum.
\end{exerc}

\newpage


%%%%%%%%%%%%%%%%%%%%%%%%%%%%%%%%%%%%%%%%%%%%%%%%%%
\section{Introduction to Lebesgue measure}
%%%%%%%%%%%%%%%%%%%%%%%%%%%%%%%%%%%%%%%%%%%%%%%%%%

\begin{defn}
  Let $A$ be any subset of $\RR^n$. The \emph{Lebesgue outer measure} of $A$ is
  \[m^*(A)=\inf\set{\left.\sum_1^\infty \vol(B_i)\;\right|\;\text{$B_i$ are boxes and }A\subset\bigcup_1^\infty B_i}
  \]
\end{defn}

\begin{defn}
  Let $A$ be any subset of $\RR^n$. We say that $A$ is \emph{Lebesgue measurable} if for every $\epsilon>0$ there exists a sequence of boxes $B_i$ such that $A\subset\bigcup B_i$ and $m^*(\bigcup B_i\setminus A)<\epsilon$. When this is the case, we define $m(A)=m^*(A)$ to be the \emph{Lebesgue measure} of $A$.
\end{defn}

Before we begin working to establish all these claims, we study the Lebesgue outer measure further.
In order to proceed, it is useful to lay out what properties are expected of an outer measure.
The following will be referred to as the \emph{outer measure axioms}.
\begin{enumerate}
\item (empty set) $m^*(\emptyset)=0$
\item (monotonicity) If $A\subset B$ then $m^*(A)\leq m^*(B)$
\item (countable subadditivity) $m^*(\bigcup A_n)\leq\sum m^*(A_n)$
\end{enumerate}
\begin{prop}
  The Lebesgue outer measure satisfies the outer measure axioms (a)--(c).
\end{prop}

\begin{proof}
  The axioms (a) and (b) are both trivial, so it remains to prove only axiom (c). Let $E_n$ be arbitrary sets and let $\epsilon>0$ be given. From the definition of Lebegue outer measure, for each $n$ we can find a sequence of boxes $B_i^n$ such that $A_n\subset\bigcup_iB_i^n$ and $\sum_im(B_i^n)-m^*(A_n)<\epsilon/2^n$.

  Taking unions, we have $\bigcup A_n\subset\bigcup_n\bigcup_iB_i^n$, and moreover:
  \begin{align*}
    m^*(\bigcup A_n)&&\leq\sum_n\sum_i\vol(B_i^n)
                    &&\leq\sum_n\left(m^*(A_n)+\epsilon/2^n\right)
                    &&\leq\sum_nm^*(A_n)+2\epsilon
  \end{align*}
  Taking $\epsilon\to0$, we obtain the desired inequality $m^*(\bigcup A_n)\leq\sum m^*(A_n)$.
\end{proof}


% \begin{exerc}[Tao, Ex 1.2.1]
%   Show that the countable union of Jordan measurable sets need not be Jordan measurable, even when bounded. Show that the countable intersection of Jordan measurable sets need not be Jordan measurable.
% \end{exerc}

% \begin{exerc}[Tao, Ex 1.2.2]
%   Give an example of a sequence of uniformly bounded, Riemann integrable functions on $[0,1]$ which converges pointwise to a function that is not Riemann integrable. Is it possible to give an example which converges uniformly?
% \end{exerc}

% \begin{exerc}
%   Show that $m^*(A)\leq m^{*j}(A)$. Give an example of a set $A$ such that $m^*(A)<m^{*j}(A)$.
% \end{exerc}

% \begin{exerc}
%   Show that Jordan outer measure does not satisfy countable subadditivity.
% \end{exerc}

% \begin{exerc}
%   Show that if $A$ is Lebesgue null, that is, $m^*(A)=0$, then $A$ is Lebesgue measurable.
% \end{exerc}

% Show that the Jordan inner measure is the same if you use infinitely many sets.


%%%%%%%%%%%%%%%%%%%%%%%%%%%%%%%%%%%%%%%%%%%%%%%%%%
\section{Lebesgue outer measure}
%%%%%%%%%%%%%%%%%%%%%%%%%%%%%%%%%%%%%%%%%%%%%%%%%%


\begin{lem}
  \label{lem:separated}
  Suppose that $A,B$ are positively separated, that is, that 
  \[d(A,B)=\inf\set{d(x,y)\mid x\in A, y\in B}>0.\]
  Then $m^*(A\cup B)=m^*(A)+m^*(B)$.
\end{lem}

\begin{thm}
  Suppose $B_i$ is a sequence of pairwise almost disjoint boxes. Then $m^*(\bigcup B_i)=\sum\vol(B_i)$.
\end{thm}

\begin{prop}
  \label{prop:open-decomp}
  Any open set $O$ can be written as a union of a sequence of pairwise almost disjoint boxes.
\end{prop}



\begin{lem}
  Let $A$ be any subset of $\RR^n$. Then $m^*(A)=\inf\set{m^*(O)\mid\text{$O$ is open and }A\subset O}$.
\end{lem}



% \begin{exerc}[Tao, Ex 1.2.5]
%   Suppose $A$ is expressible as a countable union of pairwise almost disjoint boxes. Show that $m^*(A)=m_{*j}(A)$.
% \end{exerc}

\begin{exerc}[Tao, Ex 1.2.6]
  Show that it is not true in general that
  \[m^*(A)=\sup\set{m^*(O)\mid\text{$O$ is open and }O\subset A}
  \]
\end{exerc}



%%%%%%%%%%%%%%%%%%%%%%%%%%%%%%%%%%%%%%%%%%%%%%%%%%
\section{Lebesgue measurability}
%%%%%%%%%%%%%%%%%%%%%%%%%%%%%%%%%%%%%%%%%%%%%%%%%%

\begin{thm}
  Open and closed sets are Lebesgue measurable. Complements, countable unions, and countable intersections of measurable sets are measurable.
\end{thm}

\begin{prop}
  The collection of Lebesgue measurable sets is the least $\sigma$-algebra containing both the open sets and the Lebesgue null sets.
\end{prop}

\begin{lem}
  A set $A$ is Lebesgue measurable if and only if for all $\epsilon>0$ there exists an open set $O$ such that $m^*(O\triangle A)<\epsilon$.
\end{lem}

\begin{lem}
  A set $A$ is Lebesgue measurable with finite Lebesgue measure if and only if for all $\epsilon>0$ there exists an elementary set $E$ such that $m^*(E\triangle A)<\epsilon$.
\end{lem}

\begin{exerc}[See Tao, Ex 1.2.7]
  \begin{enumerate}
    \item Show that $A$ is measurable iff for all $\epsilon>0$ there exists a closed set $F\subset A$ such that $m^*(A\setminus F)<\epsilon$.
    \item Show that $A$ is measurable iff for all $\epsilon>0$ there exists a measurable set $B$ such that $m^*(A\triangle B)<\epsilon$.
  \end{enumerate}
\end{exerc}

\begin{exerc}[Tao, Ex 1.2.14]
  Show that any set $A$ is contained in a Lebesgue measurable set $B$ such that $m(B)=m^*(A)$.
\end{exerc}

% Note: possibly needs the MCT which is in the next section!
\begin{exerc}[Tao, Ex 1.2.15]
  Show the \emph{inner regularity} property: If $A$ is Lebesgue measurable, then
  \[m(A)=\sup\set{m(K)\mid K\subset A,\text{ $K$ compact}}
  \]
\end{exerc}


%%%%%%%%%%%%%%%%%%%%%%%%%%%%%%%%%%%%%%%%%%%%%%%%%%
\section{Lebesgue measure}
%%%%%%%%%%%%%%%%%%%%%%%%%%%%%%%%%%%%%%%%%%%%%%%%%%


\begin{thm}
  The Lebesgue measure satisfies the axioms
  \begin{enumerate}
  \item (normality) if $B$ is a box then $m(B)=\vol(B)$;
  \item (translation-invariance) $m(x+A)=m(A)$ for every $x\in\RR^n$ and measurable set $A$;
  \item (countable additivity) $m(\bigcup A_n)=\sum m(A_n)$ for every sequence of pairwise disjoint measurable sets $A_n$.
  \end{enumerate}
\end{thm}

\begin{thm}
  \label{thm:mct}
  \begin{itemize}
  \item (upwards monotone convergence theorem) If $A_n$ are measurable and\\ $A_{n+1}\supset A_n$ then $m(\bigcup A_n)=\lim m(A_n)$.
  \item (downwards monotone convergence theorem) If $A_n$ are measurable and\\ $A_{n+1}\subset A_n$ then $m(\bigcap A_n)=\lim m(A_n)$, provided some $A_n$ has finite measure.
  \end{itemize}
\end{thm}

% \begin{exerc}[Tao Ex 1.2.11(iii)]
%   Give a counterexample showing that the hypothesis that some $A_n$ has finite measure is necessary for the downwards MCT.
% \end{exerc}

\begin{exerc}[Tao Ex 1.2.12]
  Suppose you know that the domain of $m$ is a $\sigma$-algebra, and $m$ satisfies $m(\emptyset)=0$ and the countable additivity property. Show that $m$ satisfies the monotonicity property and the countable subadditivity property.
\end{exerc}

\begin{exerc}[Tao Ex 1.2.13]
  Let us say that a sequence of sets $A_n$ converges to $A$ if the characteristic functions $\chi_{A_n}$ converge pointwise to $\chi_A$.
  \begin{enumerate}
    \item Show that if $A_n$ are Lebesgue measurable and $A_n$ converges to $A$ then $A$ is Lebesgue measurable. [Hint: Show that if $A_n$ converges to $A$ then $A=\bigcup_n\bigcap_{m>n}A_m$ and also $A=\bigcap_n\bigcup_{m>n}A_m$.]
    \item Suppose that if $A_n$ are all contained in a set of finite measure and $A_n$ converges to $A$, then $m(A_n)\to m(A)$. This is an example of the \emph{dominated convergence theorem}.
    \item Give a counterexample showing that the hypothesis that $A_n$ are all contained in a set of finite measure cannot be replaced with the hypothesis that the values $m(A_n)$ are bounded.
  \end{enumerate}
\end{exerc}

% %%%%%%%%%%%%%%%%%%%%%%%%%%%%%%%%%%%%%%%%%%%%%%%%%%
% %%%%%%%%%%%%%%%%%%%%%%%%%%%%%%%%%%%%%%%%%%%%%%%%%%
% \chapter{Measure and integration}
% %%%%%%%%%%%%%%%%%%%%%%%%%%%%%%%%%%%%%%%%%%%%%%%%%%
% %%%%%%%%%%%%%%%%%%%%%%%%%%%%%%%%%%%%%%%%%%%%%%%%%%

%%%%%%%%%%%%%%%%%%%%%%%%%%%%%%%%%%%%%%%%%%%%%%%%%%
\section{Preview of integration, simple integration}
%%%%%%%%%%%%%%%%%%%%%%%%%%%%%%%%%%%%%%%%%%%%%%%%%%

% Start with bounded, bounded support, simple functions and simple integrals, and then bounded, bounded support measurable functions and their integrals? Or maybe this only makes sense when Darboux has been covered (and not moved to appendix).

\begin{defn}
  A function $f$ mapping $\RR^d$ into the extended real numbers $[0,\infty]$ (or sometimes into $\CC$) is called \emph{simple} if there exists a partition of $\RR^d$ into finitely many Lebesgue measurable subsets $A_1,\ldots,A_k$ such that $f$ takes a constant value $c_i$ on each $A_i$.
\end{defn}

\begin{defn}
  If $f=\sum_1^kc_i\chi_{A_i}$ is a simple function and $f\geq0$, then the \emph{simple integral} of $f$ is defined to be $\sint f=\sum_1^kc_im(A_i)$.
\end{defn}

\begin{lem}
  If $f=\sum_1^lc_i\chi_{A_i}$ and $f=\sum_1^md_j\chi_{B_j}$ then we have $\sum_1^l c_i m(A_i)=\sum_1^m d_j m(B_j)$.
\end{lem}

\begin{prop}
  Let $f,g$ be simple functions.
  \begin{enumerate}
  \item (equivalence) if $f=g$ almost everywhere then $\sint f=\sint g$
  \item (monotonicity) if $f\leq g$ almost everywhere then $\sint f\leq\sint g$.
  \item (linearity) $\sint (cf)=c\cdot\sint f$, and $\sint (f+g)=\sint f+\sint g$.
  \end{enumerate}
\end{prop}


% Could consider moving the above paragraph to the corresponding section

\begin{exerc}
  Show that a function $f$ is simple if and only if it can be expressed as $f=\sum_1^kc_i\chi_{A_i}$, where $A_i$ are (not necessarily disjoint) Lebesgue measurable sets.
\end{exerc}

\begin{exerc}[see Tao, Ex 1.3.1]
  Show that the simple integral satisfies the properties:
  \begin{enumerate}
    \item (finiteness) $\sint f<\infty$ if and only if $f$ is finite almost everywhere and supported on a set of finite measure
    \item (vanishing) $\sint f=0$ if and only if $f=0$ almost everywhere
    \item (normality) $\sint\chi_A=m(A)$ for any Lebesgue measurable set $A$
  \end{enumerate}
\end{exerc}


%%%%%%%%%%%%%%%%%%%%%%%%%%%%%%%%%%%%%%%%%%%%%%%%%%
\section{Lebesgue integration of nonnegative functions}
%%%%%%%%%%%%%%%%%%%%%%%%%%%%%%%%%%%%%%%%%%%%%%%%%%



\begin{defn}
  Let $f$ be a nonnegative function on $\RR^d$. We define the \emph{lower Lebesgue integral} of $f$ by
  \[\underline{\int}f=\sup\left\{\left.\text{s}\kern-2pt\int g\;\right|\; g\leq f,\text{ $g$ nonnegative simple}\right\}
  \]
  If $f$ is measurable, we define the \emph{Lebesgue integral} of $f$ to be $\int f=\lint f$.
\end{defn}

\begin{prop}
  The lower Lebesgue integral satisfies the properties:
  \begin{enumerate}
  \item (equivalence) if $f=g$ almost everywhere then $\lint f=\lint g$;
  \item (monotonicity) if $f\leq g$ almost everywhere then $\lint f\leq\lint g$; and
  \item (superadditivity) $\lint(f+g)\geq\lint f+\lint g$.
  \end{enumerate}
\end{prop}

\begin{lem}
  Let $f$ be a nonnegative function on $\RR^d$. The lower Lebesgue integral satisfies the following identities.
  \begin{enumerate}
  \item (range truncation) If $f^N=\min(f,N)$ then $\lint f^N\to\lint f$.
  \item (support trunctation) If $f_N=f\chi_{[-N,N]^d}$ then $\lint f_N\to\lint f$.
  \end{enumerate}
\end{lem}

\begin{thm}
  If $f,g$ are nonnegative measurable functions, then $\int(f+g)=\int f+\int g$.
\end{thm}

\begin{exerc}[Tao, ex 1.3.13]
  Let $f$ be a nonnegative measurable function on $\RR$. Show that $\int f$ is equal to the $2$-dimensional Lebesgue measure of the region
  \[\set{(x,y)\mid 0\leq y\leq f(x)}
  \]
\end{exerc}

\begin{exerc}[Tao, ex 1.3.18]
  Let $f$ be an nonnegative measurable function on $\RR^d$.
  \begin{enumerate}
    \item Show that if $\int f<\infty$ then $f$ is finite almost everywhere. Give a counterexample to show that the converse is false.
    \item Show that $\int f=0$ if and only if $f=0$ almost everywhere.
  \end{enumerate}
\end{exerc}

\begin{exerc}
  Give an example of a nonnegative function which is measurable, but has different lower and upper Lebesgue integrals.
\end{exerc}


%%%%%%%%%%%%%%%%%%%%%%%%%%%%%%%%%%%%%%%%%%%%%%%%%%
\section{Lebesgue measurability of functions}
%%%%%%%%%%%%%%%%%%%%%%%%%%%%%%%%%%%%%%%%%%%%%%%%%%



\begin{defn}
  A nonnegative function $f$ on $\RR^n$ is said to be a \emph{measurable function} if $f$ is the pointwise limit of nonnegative simple functions.
\end{defn}

\begin{thm}
  \label{thm:measurable-equiv}
  A nonnegative function $f$ is measurable if and only if either of the following holds.
  \begin{enumerate}
  \item there is a sequence $f_n$ of simple functions such that the $f_n$ are bounded and have bounded support, the $f_n$ are increasing $f_n\leq f_{n+1}$, and $f=\sup f_n$;
  \item for any open set $S$ (respectively: closed set, interval, ray, etc) the preimage $f^{-1}(S)$ is Lebesgue measurable.
  \end{enumerate}
\end{thm}

It is worth remarking that by property (b) of the Lemma, measurability can be viewed as a massive generalization of continuity. Recall that a function $f$ is continuous if and only if whenever $S$ is open we have $f^{-1}(S)$ open. In property (b), we ask merely that $f^{-1}(S)$ be Lebesgue measurable, a much weaker demand.

Notice also that since preimages are stable under unions, intersections, and complements, property (b) implies that if $S$ is Borel then $f^{-1}(S)$ will be measurable too. But if $S$ is merely measurable, there is no guarantee that $f^{-1}(S)$ will be measurable! To see this consider a function $f$ which is a bijection between $[0,1]$ and a null set. For example one can map $[0,1]$ into the Cantor set $C$ injectively almost everywhere by operating on binary and ternary expansions as follows:
\[0.b_1b_2b_3\cdots\text{ (base 2)}\quad\mapsto\quad 0.(2b_1)(2b_2)(2b_3)\cdots\text{ (base 3)}
\]
Now if $N$ is a Lebesgue nonmeasurable subset of $[0,1]$, we have that $S=f(N)$ is null but the preimage $f^{-1}(S)$ is non-measurable.

To close the section, we extended the definition of measurable function from nonnegative functions only to complex-valued functions in the following way. Recall that if $f$ is a real-valued function, then we can define its \emph{positive and negative parts}:
\begin{align*}
  f^+&=\max(f,0)&
  f^-&=\max(-f,0)
\end{align*}
We then have that $f^+$ and $f^-$ are nonnegative functions with $f=f^+-f^-$.

\begin{defn}
  If $f$ is an almost-everywhere defined complex-valued function on $\RR^n$ then $f$ is a \emph{measurable function} if and only if the positive and negative parts of its real and imaginary parts are measurable functions.
\end{defn}

\begin{lem}
  If $f$ is a complex-valued function on $\RR^d$ then $f$ is measurable if and only if $f$ is a pointwise limit of complex-valued simple functions.
\end{lem}

% \begin{exerc}
%   Show that $\lim x_n=x$ if and only if $\liminf x_n=x=\limsup x_n$.
% \end{exerc}

\begin{exerc}[Tao, Ex 1.3.4]
  Show that if $f$ is a bounded, nonnegative measurable function on $\RR^d$, then there is a sequence of bounded simple functions $f_n$ which converges uniformly to $f$ (not just pointwise).
\end{exerc}

\begin{exerc}[Tao, Ex 1.3.5]
  Let $f$ be a nonnegative function on $\RR^d$. Show that $f$ is simple if and only if $f$ is measurable and takes on at most finitely many values.
\end{exerc}

\begin{exerc}[Tao, Ex 1.3.6]
  If $f$ is a nonnegative measurable function, show that the region under $f$ is a measurable set.
\end{exerc}


%%%%%%%%%%%%%%%%%%%%%%%%%%%%%%%%%%%%%%%%%%%%%%%%%%
\section{Lebesgue integration}
%%%%%%%%%%%%%%%%%%%%%%%%%%%%%%%%%%%%%%%%%%%%%%%%%%


\begin{defn}
  Let $f$ be a complex-valued measurable function on $\RR^d$ (it need only be defined almost everywhere). We say that $f$ is \emph{absolutely integrable}, or a member of $L^1$, if $\int|f|<\infty$.
\end{defn}

\begin{defn}
  If $f$ is absolutely integrable and real-valued, let $\int f=\int f^+-\int f^-$. If $f$ is absolutely integrable and complex-valued, let $\int f=\int\Re f+i\int\Im f$.
\end{defn}

%% monotonicity?

%% next result: should be moved to the previous section, but then referenced in the next proposition as an inherited property

\begin{prop}
  Let $f\colon[a,b]\to\RR$ be a Riemann integrable function. Then intepreting $f$ as a function defined on all of $\RR$ which is zero outside of $[a,b]$, we have that $f$ is Lebesgue absolutely integrable with the same value.
\end{prop}


\begin{prop}
  \begin{enumerate}
  \item (linearity and conjugation) $\int(f+g)=\int f+\int g$, $\int cf=c\int f$, and $\int\bar f=\overline{\int f}$;
  \item (triangle inequality for integrals) $\left|\int f\right|\leq\int|f|$.
  \end{enumerate}
\end{prop}

\begin{prop}
  The collection of absolutely integrable functions forms a vector space. In fact it is a seminormed vector space with the seminorm $\|f\|=\int|f|$.
\end{prop}

\begin{thm}
  \label{thm:density}
  The following are all dense subsets of the space of absolutely integrable functions.
  \begin{enumerate}
  \item absolutely integrable simple functions;
  \item absolutely integrable simple functions $\sum_1^kc_i\chi_{B_i}$ where $B_i$ are all boxes; and
  \item continuous, compactly supported functions.
  \end{enumerate}
\end{thm}

\begin{cor}
  Given \(f\) absolutely integrable  and \(\epsilon>0\)
  \begin{enumerate}
  \item \(\exists g\) simple integrable s.t. \(\int|f-g|<\epsilon\) (symmetric triangle inequality)
  \item \(\exists h\) boxwise const s.t. \(\int|f-h|<\epsilon\)
  \item \(\exists \psi\) continuous with \(\psi\equiv0\) on complement of a bdd set s.t.  \(\int|f-\psi|<\epsilon\)
  \end{enumerate}
\end{cor}

\begin{thm}[Egorov]
  Measurable \(f_n\to f\) pointwise a.e., then \(\exists B\) meas. (bad set) with \(m(B)<\epsilon\) s.t. \(f_n\to f\) uniformly on each bdd subset of \(\RR\backslash B\).
\end{thm}

\begin{thm}[Lusin]
  Measurable \(f\) absolutely integrable, \(\epsilon>0\), then \(\exists B\) meas. (bad set) with \(m(B)<\epsilon\) s.t. \(f|_{\RR^d\backslash B}\) is a continuous function on \(\RR^d\backslash B\).
  (can choose \(B\) open and \(\RR^d\backslash B\) closed).
\end{thm}


\begin{exerc}[Tao, ex 1.3.25(i)]
  Let $f$ be absolutely integrable. Show that for any $\epsilon>0$ there exists a bounded measurable set $A$ such that $\int |f|\chi_{A^c}<\epsilon$.
\end{exerc}

\begin{exerc}[Tao, ex 1.3.25(ii)]
  Let $f$ be a nonnegative measurable function, and assume $f$ is finite almost everywhere. Show that for any $\epsilon>0$, there exists a measurable set $A$ such that $m(A)<\epsilon$ and $f$ is locally bounded outside of $A$. In other words, for every $n$ there exists $M$ such that for all $x\in[-n,n]^d\setminus A$ we have $f(x)\leq M$.
\end{exerc}

\begin{exerc}
  Show that the space of absolutely integrable functions is separable, that is, has a countable dense subset.
\end{exerc}


%%%%%%%%%%%%%%%%%%%%%%%%%%%%%%%%%%%%%%%%%%%%%%%%%%
\section{Abstract measure theory}
%%%%%%%%%%%%%%%%%%%%%%%%%%%%%%%%%%%%%%%%%%%%%%%%%%

The Lebesgue measure and integration theory that we have developed can be regarded as a model for an abstract concept of a measure and integral. The situation is very similar to other areas of mathematics. Consider the following examples of a concrete and abstract concept:
\begin{itemize}
\item the space $\RR^d$ with its distance measurement $\|x-y\|$ is a model for the definition of metric space;
\item the space $\RR^d$ with its family of open sets is a model for the definition of topological space;
\item the spaces $\RR$ or $\CC$ with their addition and multiplication operations are models for the definition of field;
\item the space $\RR^d$ with its $\RR$-linear combinations is a model for the definiton of vector space.
\end{itemize}

\begin{defn}
  A \emph{measurable space} is a pair $(X,\mathcal B)$ where $X$ is any set and $\mathcal B$ is a Boolean \emph{$\sigma$-algebra} on $X$: a collection of subsets of $X$ which contains the sets $\emptyset$ and $X$, and is closed under countable intersections, countable unions, and complements.
\end{defn}

\begin{defn}
  Suppose $X$ is a set and $\mathcal B$ is a $\sigma$-algebra on $X$. A function $\mu\colon\mathcal B\to[0,\infty]$ is said to be a \emph{measure} if it satisfies
  \begin{enumerate}
  \item (empty set) $\mu(\emptyset)=0$; and
  \item (countable additivity) for every sequence of pairwise disjoint sets $A_n\in\mathcal B$, we have $\mu(\bigcup A_n)=\sum(\mu(A_n))$.
  \end{enumerate}
\end{defn}

\begin{prop}
  Suppose that $\mathcal B$ is a $\sigma$-algebra on $X$, and $\mu$ is a measure on $\mathcal B$. Let $A,B$, and $A_n$ denote elements of $\mathcal B$. Then we have:
  \begin{itemize}
    \item (monotonicity) if $A\subset B$ then $\mu(A)\leq\mu(B)$;
    \item (inclusion--exclusion) $\mu(A\cup B)+\mu(A\cap B)=\mu(A)+\mu(B)$.
    \item (countable subadditivity) $\mu(\bigcup A_n)\leq\sum\mu(A_n)$;
    \item (upwards monotone convergence) if $A_{n+1}\supset A_n$ then $\mu(\bigcup A_n)=\lim\mu(A_n)$; and
    \item (downwards monotone convergence) if $A_{n+1}\subset A_n$ and $\mu(A_1)<\infty$ then $\mu(\bigcap A_n)=\lim\mu(A_n)$.
  \end{itemize}
\end{prop}

\begin{defn}
  A \emph{Boolean algebra} on $X$ is a collection of subsets of $X$ which contains the sets $\emptyset$, and $X$, and is closed under pairwise intersections, pairwise unions, and complements.
\end{defn}

\begin{defn}
  Suppose $X$ is a set and $\mathcal A$ is a Boolean algebra on $X$. A function $\mu\colon\mathcal A\to[0,\infty]$ is said to be a \emph{finitely additive measure} if it satisfies
  \begin{enumerate}
  \item (empty set) $\mu(\emptyset)=0$; and
  \item (finite additivity) for all disjoint sets $A,B\in\mathcal A$, we have $\mu(A\cup B)=\mu(A)+\mu(B)$.
  \end{enumerate}
\end{defn}

\begin{defn}
  Let $X$ be any set, $\mathcal A$ a Boolean algebra on $X$, and $\mu_0$ a finitely additive measure on $\mathcal A$. Then $\mu_0$ is said to be a \emph{premeasure} if it satisfies the additional axiom:
  \begin{itemize}
  \item for every sequence of pairwise disjoint sets $A_n\in\mathcal A$, if $\bigcup A_n\in\mathcal A$, then $\mu_0(\bigcup A_n)=\sum\mu(A_n)$.
  \end{itemize}
\end{defn}

\begin{thm}[Consequence of Carath\'eodory's extension theorem]
  Let $X$ be a set, $\mathcal A$ an algebra on $X$, and $\mu_0$ a premeasure on $\mathcal A$. Let $\mathcal B$ be the $\sigma$-algebra generated by $\mathcal A$. Then $\mu_0$ extends to a measure $\mu$ on $\mathcal B$.
\end{thm}

\begin{defn}
  Let $X$ be any set and $\mu^*$ a function on all subsets of $X$. Then $\mu^*$ is said to be a \emph{outer measure} if it satisfies
  \begin{enumerate}
  \item (empty set) $\mu^*(\emptyset)=0$;
  \item (monotonicity) if $A\subset B$ then $\mu^*(A)\leq\mu^*(B)$; and
  \item (countable subadditivity) $\mu^*(\bigcup A_n)\leq\sum\mu^*(A_n)$.
  \end{enumerate}
\end{defn}

If $\mathcal A$ is an algebra on $X$, and $\mu_0$ is any function on $\mathcal A$ which satisfies the empty set axiom, then we can define
\begin{equation}
  \label{eq:outer}
  \mu^*(E)=\inf\left\{\left.\sum\mu_0(A_i)\;\right|\;A_i\in\mathcal A,\text{ and }E\subset\bigcup A_i\right\}
\end{equation}

\begin{prop}
  If $\mu^*$ is constructed as in Equation~\eqref{eq:outer}, then $\mu^*$ is an outer measure.
\end{prop}


\begin{exerc}[Tao, ex 1.4.26]
  Let $\mu$ be a measure on $(X,\mathcal B)$. Show that $\mathcal B$ can be extended to $\sigma$-algebra $\hat{\mathcal B}$ and $\mu$ to a measure $\hat\mu$ on $(X,\hat{\mathcal B})$ in such a way that $\hat\mu$ is complete.
\end{exerc}

\begin{exerc}[Tao, ex 1.4.49]
  Let $f$ be a nonnegative Lebesgue measurable function. Show that $\mu(A)=\int f\chi_A$ is a measure.
\end{exerc}

% Tao 1.7.4

% \begin{exerc}[Tao, ex 1.7.6]
%   Give an example of a finitely additive measure that is not a premeasure. [Hint: work on the measurable space $(\NN,\mathcal P(\NN))$ and define $\mu_0$ separately for finite and infinite sets.]
% \end{exerc}


%%%%%%%%%%%%%%%%%%%%%%%%%%%%%%%%%%%%%%%%%%%%%%%%%%
\section{Convergence theorems}
%%%%%%%%%%%%%%%%%%%%%%%%%%%%%%%%%%%%%%%%%%%%%%%%%%

\begin{example}[Domain escape to infinity]
  Let $f_n=\chi_{[n,n+1]}$ and $f=0$. That is, $f_n$ is a sequence of moving unit bumps. Then $f_n\to f$ pointwise (not uniformly), but we have $\int f_n=1$ for all $n$, and $\int f=0$.
\end{example}

\begin{example}[Support escape to infinity]
  Let $f_n=\frac1n\chi_{[0,n]}$ and $f=0$. That is, $f_n$ is a sequence of widening and shortening bumps. Then $f_n\to f$ uniformly, but once again we have $\int f_n=1$ for all $n$, and $\int f=0$.
\end{example}

\begin{example}[Range escape to infinity]
  Let $f_n=n\chi_{[1/n,2/n]}$ and $f=0$. That is, $f_n$ is a sequence of narrowing and tallening bumps. Then $f_n\to f$ pointwise (not uniformly), but once again we have $\int f_n=1$ for all $n$, and $\int f=0$.
\end{example}

\begin{thm}[Fatou's lemma]
  Let $f_n$ be nonnegative measurable functions. Then
  \[\int\liminf f_n\leq\liminf\int f_n
  \]
  In particular, if $f_n\to f$ then we have $\int f\leq\liminf\int f_n$.
\end{thm}

\begin{thm}[Dominated convergence theorem]
  Let $f_n$ be a sequence of measurable complex-valued functions and suppose that $f_n\to f$. Suppose that there exists a nonnegative function $G$ such that $\int|G|<\infty$ and for all $n$, we have $|f_n|\leq G$. Then $\int f_n\to \int f$.
\end{thm}

\begin{thm}[Monotone convergence theorem]
  Let $f_n$ be a sequence of nonnegative measurable functions, and suppose that $f_n\leq f_{n+1}$ for all $n$. Let $f=\sup f_n$, so that we automatically have $f_n\to f$. Then $\int f_n\to\int f$.
\end{thm}


It is worth remarking that the monotone convergence theorem can fail for signed functions $f_n$. As a silly example, if $f_n=-1/n$ then $f_n\to0$, $\int f_n=-\infty$ but $\int f=0$. The monotone convergence theorem has the following easy but important consequence that the nonnegative Lebesgue integral is \emph{countably} linear, not just finitely linear!

\begin{cor}[Tonelli's theorem]
  If $f_n$ is a sequence of nonnegative measurable functions, then $\int\sum f_n=\sum\int f_n$.
\end{cor}

\begin{exerc}[Tao, ex 1.44, 1.45]
  \begin{enumerate}
    \item Let $A_n$ be measurable sets and assume that\\ $\sum m(A_n)<\infty$. Show that almost every $x\in\RR^d$ is contained in at most finitely many of the $A_n$. [Hint: Use Tonelli's theorem on the functions $\chi_{A_n}$.] This is the Borel--Cantelli lemma.
    \item Give a counterexample to the above conclusion, showing that the hypothesis $\sum m(A_n)<\infty$ cannot be replaced by the weaker condition $\lim m(A_n)=0$.
  \end{enumerate}
\end{exerc}

\begin{exerc}
  Use the dominated convergence theorem to show that the harmonic series $\sum\frac1n$ diverges. [Hint: Let $f_n=\frac1n\chi_{[0,n]}$, show that $\sum\frac1n<\infty$ plus the dominated convergence theorem implies $\int f_n\to 0$, and obtain a contradiction from this.]
\end{exerc}


%%%%%%%%%%%%%%%%%%%%%%%%%%%%%%%%%%%%%%%%%%%%%%%%%%
\section{Abstract integration theory}
%%%%%%%%%%%%%%%%%%%%%%%%%%%%%%%%%%%%%%%%%%%%%%%%%%


\begin{defn}
  Suppose that $(X,\mathcal B)$ is a measurable space. A function $f$ which maps $X$ to $[0,\infty]$, $\RR$, or $\CC$ is called \emph{measurable} if the pre-image of any open set is in $\mathcal B$, that is, for every open subset $U$ of the codomain, $f^{-1}(U)\in\mathcal B$.
\end{defn}

\begin{prop}
  The measurable functions on $(X,\mathcal B)$ are closed under sums, products, continuous compositions, and pointwise limits.
\end{prop}



\begin{defn}
  Let $(X,\mathcal B)$ be a measurable space. A function $f\colon X\to[0,\infty]$ is said to be \emph{simple} if it is measurable and takes finitely many values, that is, if $f=\sum_1^kc_i\chi_{A_i}$ where $A_i\in\mathcal B$.

  If $f$ is the simple function above, and $\mu$ is a measure on $(X,\mathcal B)$, then we define the \emph{simple integral} of $f$ with respect to $\mu$ by
  \[\sint f\,d\mu=\sum_1^kc_i\mu(A_i)
  \]
\end{defn}

It is possible to check, as we have done in the case of Lebesgue integration, that the simple integral is well-defined, linear, and has numerous other familiar properties.

\begin{defn}
  Suppose $(X,\mathcal B)$ is a measurable space, and $\mu$ is a measure on it. If $f\colon X\to[0,\infty]$ is measurable, then we define the \emph{integral of $f$ with respect to $\mu$} by
  \[\int f\,d\mu=\sup\left\{\left.\sint g\,d\mu\;\right|\;g\leq f\text{, $g$ simple}\right\}
  \]
\end{defn}

\begin{prop}
  The integral of nonnegative measurable functions satisfies the following properties
  \begin{enumerate}
  \item (equivalence) if $f=g$ almost everywhere then $\int f\,d\mu=\int g\,d\mu$;
  \item (monotonicity) if $f\leq g$ almost everywhere then $\int f\,d\mu\leq\int g\,d\mu$;
  \item (range truncation) if $f^N=\min(f,N)$ then $\int f^N\,d\mu\to\int f\,d\mu$; and
  \item (support truncation) if $A_n$ is a sequence of measurable sets such that $A_n\subset A_{n+1}$ and $\bigcup A_n=X$, then $\int f\chi_{A_n}\,d\mu\to\int f\,d\mu$.
  \item (Markov's inequality) if $f$ is measurable, then $\mu(\{x\mid f(x)\geq\lambda\})\leq\frac1\lambda\int f\,d\mu$;
  \item (additivity) $\int(f+g)\,d\mu=\int f\,d\mu+\int g\,d\mu$;
  \end{enumerate}
\end{prop}

\begin{defn}
  Suppose that $(X,\mathcal B)$ is a measurable space and $\mu$ is a measure on it. If $f\colon X\to\CC$ is measurable and defined $\mu$-almost everywhere, we say that $f$ is \emph{absolutely integrable with respect to $\mu$} if $\int|f|\,d\mu<\infty$.

  In this case, we define $\int f\,d\mu=\int\Re f\,d\mu+i\int\Im f\,d\mu$, where for real-valued functions $f$ we define $\int f\,d\mu=\int f^+\,d\mu-\int f^-\,d\mu$.
\end{defn}

\begin{prop}
  The space $L^1(\mu)$ together with the norm $\|\cdot\|$ is a normed vector space.
\end{prop}

\begin{exerc}[See Tao, ex 1.4.29]
  \begin{itemize}
    \item Show that $f$ is measurable if and only if $f^+$ and $f^-$ are measurable.
    \item Show that sums and products of measurable functions are measurable.
  \end{itemize}
\end{exerc}

\begin{exerc}[See Tao, ex 1.4.36]
  Establish the support truncation property: if $A_n$ is a sequence of measurable sets such that $A_n\subset A_{n+1}$ and $\bigcup A_n=X$, then $\int f\chi_{A_n}\,d\mu\to\int f\,d\mu$.
\end{exerc}



% Additional topic: Fourier analysis?

% Additional topic: Haar measure?

\end{document}
