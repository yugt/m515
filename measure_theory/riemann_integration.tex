
%%%%%%%%%%%%%%%%%%%%%%%%%%%%%%%%%%%%%%%%%%%%%%%%%%
\section{Riemann integration}
%%%%%%%%%%%%%%%%%%%%%%%%%%%%%%%%%%%%%%%%%%%%%%%%%%

% Move to appendix?

\begin{reading}
  Tao, \S1.1.3.
\end{reading}

If the picture of Lemma~\ref{lem:jordan-graph} reminded you of Riemann sums, it should. Measure theory is closely connected to integration theory, as both are concerned with calculating areas of some regions. Moreover the Jordan measure corresponds neatly with the Riemann integral. The following presentation of the Riemann integral is actually attributed to Darboux.

Just as we defined the elementary measure before we defined the Jordan measure, we will now define the ``piecewise constant'' integral before we define the Riemann integral.

\begin{defn}
  Let $f$ be a real-valued function defined on $[a,b]$. Then $f$ is said to be \emph{piecewise constant} if there exists a partition $\mathcal P$ of $[a,b]$ into finitely many subintervals $I_j$ such that $f$ takes a constant value $c_j$ on each interval $I_j$.
\end{defn}

In other words, $f$ is piecewise constant if $f$ is of the form $\sum_1^k c_j\chi_{I_j}$, where $I_j$ are intervals. Here $\chi_{I_j}$ denotes the \emph{characteristic function} of $I_j$, that is, $\chi_{I_j}(x)=1$ if $x\in I_j$ and $\chi_{I_j}(x)=0$ otherwise.

\begin{defn}
  If $f=\sum_1^kc_j\chi_{I_j}$ then the \emph{pc integral} of $f$ is defined to be $\sum_1^kc_j\len(I_j)$.
\end{defn}

As was the case with the elementary measure, one must check that the value of the pc integral is well-defined. That is, if $f$ is expressed in two different ways as a pc function, say $\sum c_j\chi_{I_j}=\sum d_k\chi_{J_k}$, then one must check that the two values $\sum c_j\len(I_j)$ and $\sum d_k\len(J_k)$ agree.

\begin{defn}
  Let $f$ be a bounded function on $[a,b]$. First define the lower and upper Riemann forms:
  \begin{align*}
    \lint f&=\sup\set{\left.\text{pc}\!\!\int\!\! f\;\right|\;g\leq f\text{, $g$ pc}}\\
    \ovint f&=\inf\set{\left.\text{pc}\!\!\int\!\! h\;\right|\;f\leq h\text{, $h$ pc}}
  \end{align*}
  Then if $\lint f=\ovint f$ we say that $f$ is \emph{Riemann integrable}, and denote the common value simply by $\int f$.
\end{defn}

% We should probably note there is a lemma similar to the previous section showing that we can characterize integrability by for all epsilon, there exists g,h, etc. This knits the proofs below together.

\begin{prop}
  \label{prop:riemann-properties}
  The Riemann integral satisfies the three properties:
  \begin{itemize}
  \item (normality) If $A$ is a Jordan measurable subset of $[a,b]$, then $\chi_A$ is Riemann integrable over $[a,b]$ and $\int\chi_A=m(A)$.
  \item (linearity) If $f,g$ are Riemann integrable then so are $cf$ and $f+g$ and we have $\int cf=c\int f$, and $\int(f+g)=\int f+\int g$.
  \item (monotonicity) If $f,g$ are Riemann integrable and $f\leq g$ then $\int f\leq \int g$.
  \end{itemize}
\end{prop}

\begin{proof}
  We establish only the normality property. By Lemma~\ref{lem:jordan-equiv}, for any $\epsilon$ we can find disjoint intervals $I_j$ and disjoint intervals $J_k$ such that $\bigcup I_j\subset A\subset\bigcup J_k$ and $m(\bigcup J_k\setminus\bigcup I_j)<\epsilon$. It is easy to see from the definition of the pc integral that $\text{pc}\int\chi_{\bigcup I_j}=m(\bigcup I_i)$, and similarly $\text{pc}\int\chi_{\bigcup J_k}=m(\bigcup J_k)$. We now have
  \[m(\bigcup I_i)\leq\lint\chi_A\leq\ovint\chi_A
  \leq m(\bigcup J_k)
  \]
  Since the left and right-hand sides differ by $<\epsilon$, it follows that the lower and upper integrals differ by $<\epsilon$ as well. Since $\epsilon$ was arbitrary, it follows that $\chi_A$ is integrable. And since we also have
  \[m(\bigcup I_i)\leq m(A)\leq m(\bigcup J_k)
  \]
  we may conclude that $\int\chi_A$ is equal to $m(A)$.
\end{proof}

If one re-examines the definition and properties of the Jordan measure, it should be clear that there is a close parallel between the Riemann integral and Jordan measure. The normality property above begins to make this connection formal. The next result further strenghens the two-way connection between the two notions.

\begin{thm}
  If $f$ is a nonnegative, bounded function on $[a,b]$, then $f$ is Riemann integrable if and only if the region $A=\set{(x,y)\mid0\leq y\leq f(x)}$ is Jordan measurable. Moreover, in this case we have $\int f=m(A)$.
\end{thm}

\begin{proof}
  First suppose that $f$ is Riemann integrable and let $\epsilon>0$ be given. Choose pc functions $g,h$ such that $g\leq f\leq h$ and $\text{pc}\int(h-g)<\epsilon$. Let $E$ be the region under the graph of $g$ and let $F$ be the region under the graph of $h$. It is clear that $E,F$ are elementary, $E\subset A\subset F$, and $m(F\setminus E)<\epsilon$.

  Conversely if $A$ is Jordan measurable we can find an elementary $E$ such that $E\subset A$ and $m(A\setminus E)<\epsilon$. Using our usual grid argument, we can suppose that there is a sequence of disjoint intervals $I_j$ such that $E$ is a union of boxes with horizontal sides selected from the $I_j$. Pairing each $I_j$ with the constant $c_j=$ the maximum of the vertical coordinates of all of the boxes with horizontal side $I_j$, we obtain a pc function $g$. It is easy to see that $m(E)\leq \text{pc}\int g\leq m(A)$. This shows that the lower Riemann integral of $f$ is $m(A)$. We can proceed similarly using an outer approximation $B$ to show that the upper Riemann integral of $f$ is $m(A)$ too.
\end{proof}

Depending on when you last studied Riemann integration, you may better recall Riemann's classical approach rather than the Darboux approach above. This version involves a quite expansive notation:

\begin{itemize}
\item $f$ denotes a real-valued, bounded function defined on the interval $[a,b]$.
\item $x_0,x_1,\ldots,x_k$ denotes an increasing sequence of points in $[a,b]$ (they will be rectangle endpoints), where $x_0=a$ and $x_k=b$.
\item $\mathcal P$ denotes the partition of $[a,b]$ into subintervals defined by the $x_i$, that is, into subintervals $[x_{i-1},x_i]$.
\item $\delta x_i$ denotes the length of the $i$th interval, $x_i-x_{i-1}$.
\item $\|\mathcal P\|$ denotes the norm of the partition, $\max\delta x_i$.
\item $x_1^*,\ldots,x_k^*$ denotes any selection of points such that $x_i^*\in[x_{i-1},x_i]$.
\end{itemize}

With these pieces in hand, we can define the Riemann sums and the Riemann integral.

\begin{defn}
  With $f$, $\mathcal P$, $\delta x_i$, $x_i^*$ as above, the corresponding \emph{Riemann sum} is:
  \[\mathcal R(f,\mathcal  P,x_i^*)=\sum f(x_i^*)\delta x_i
  \]
  The \emph{Riemann integral} of $f$ on $[a,b]$ is then defined by
  \[R\int_a^b f=\lim_{\|\mathcal P\|\to0}\mathcal R(f,\mathcal P,x_i^*)
  \]
  provided this limit exists. Here the limit ``exists'' and equals $L$ if for all $\epsilon>0$ there exists $\delta>0$ such that for all $\mathcal P$ and $x_i^*$ we have $\|\mathcal P\|<\delta$ implies $|R(f,\mathcal P,x_i^*)-L|<\epsilon$.
\end{defn}

It is an exercise in both notation and partition management to check that $f$ is Riemann integrable in the Darboux sense described earlier in this section if and only if $f$ is Riemann integrable in the classical Riemann sense just defined.

\begin{exerc}[See Tao, Ex 1.1.21]
  Show that the pc integral is well-defined, and satisfies the normality, linearity, and monotonicity properties.
\end{exerc}

\begin{exerc}[Tao, Ex 1.1.22]
  Let $f$ be a bounded function on the interval $[a,b]$. Then $f$ is integrable in the Darboux sense if and only if $f$ is integrable in the classical Riemann sense, and in this case the two values agree.
\end{exerc}

\begin{exerc}[Tao, Ex 1.1.23]
  Let $f\colon[a,b]\to\RR$. Show that if $f$ is continuous, then $f$ is Riemann integrable. Show that if $f$ is bounded and piecewise continuous, then $f$ is Riemann integrable.
\end{exerc}

\begin{exerc}[Tao, Ex 1.1.24]
  Complete the proof of Proposition~\ref{prop:riemann-properties}: Show that the Riemann integral satisfies the linearity and monotonicity properties. (Hint: first establish these properties for the pc integral.)
\end{exerc}

\newpage